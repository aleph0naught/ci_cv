% neomake: skip
\documentclass{article}

\usepackage{cmap}
\usepackage{array}
\usepackage{enumitem}
\usepackage{makecell}
\usepackage[T2A]{fontenc}
\usepackage[utf8]{inputenc}
\usepackage[english,russian]{babel}
\usepackage{indentfirst}
\usepackage{amssymb}
\usepackage{amsmath}
\usepackage{multicol}
\usepackage{fontawesome}
\usepackage{verbatim}

\usepackage{geometry}
\geometry{top=10mm}
\geometry{bottom=10mm}
\geometry{left=15mm}
\geometry{right=15mm}

\pagenumbering{gobble}

\usepackage[dvipsnames]{xcolor}
\usepackage[colorlinks  = true,
            linkcolor   = blue,
            urlcolor    = blue,
            citecolor   = black,
            anchorcolor = black]{hyperref}

\renewcommand{\maketitle}{
    \Huge
    \begin{center}
        \textbf{Орлов Александр}
    \end{center}

    \small
    \begin{center}
    \faMobile \hspace{0.1cm} $\boldsymbol{+}$7(929)931-31-62 
    \hfill
    \faEnvelope \hspace{0.1cm} \href{mailto:orlovalex0703@gmail.com}{orlovalex0703@gmail.com}
    \hfill
    \faPaperPlane \hspace{0.1cm} \href{https://t.me/aleph0naught}{@aleph0naught}
    \hfill
    \faGithub \hspace{0.1cm} \href{https://github.com/aleph0naught}{aleph0naught}
    \end{center}
}

\usepackage{titlesec}
\titleformat{\section}{\Large\bf\raggedright}{}{0.5em}{}[{\titlerule[1pt]}]
\titlespacing{\section}{0pt}{3pt}{7pt}
\titleformat{\subsection}{\large\bfseries\raggedright}{}{0em}{\underline}%[\rule{3cm}{.2pt}]
\titlespacing{\subsection}{0pt}{7pt}{7pt}


\newcommand{\entry}[3]{
	\begin{tabular}{ c | c }
    \begin{minipage}{0.05\linewidth}
    	\begin{center}
    		#1
    	\end{center}
    \end{minipage} 
    &
    \begin{minipage}{0.85\linewidth}
        \textbf{#2} \\ \footnotesize{#3}
    \end{minipage}
    \end{tabular}
}

\newcommand{\interval}[2]{
	#1 \\ $\downarrow$ \\ #2
}

\newcommand{\specialcell}[2][c]{%
  \begin{tabular}[#1]{@{}l@{}}#2\end{tabular}}
  
  
\setlist[itemize]{noitemsep, topsep=0pt}

\begin{document}
    \maketitle
    \small
    
    \section{Образование}
        \entry {\interval{2019}{н.в.}}
        {Высшая школа экономики\\
         Факультет компьютерных наук\\
         Прикладная математика и информатика}
        {Специализация <<Машинное обучение и приложения>>\\
        Очная форма обучения, бакалавриат}

    \section{Работа}
        \entry {\interval{Июль 2021}{н.в.}}
        {Тинькофф \\
         Департамент рисков \\
         Младший аналитик}
        {Риск-аналитика:
        	\begin{itemize}
        		\item Расчет recovery для NPV моделей
        		\item Разработка ML модели оценки recovery
        		\item Создание регулярных отчетов по показателям collection
        	\end{itemize}
        }
        
        \vspace{.1cm}
        
        \entry {\interval{Апр. 2021}{Авг. 2021}}
        {Высшая школа экономики \\
         Факультет компьютерных наук \\
         Международная лаборатория биоинформатики \\
         Стажер-исследователь}
        {Биоинформатика:
        	\begin{itemize}
        		\item Исследование вторичных структур ДНК
        		\item Статистическая обработка биологических данных
        	\end{itemize}
        } 
    
    \section{Курсы}
    
    \entry {2021}
        {Тинькофф образование\\
        Анализ данных в индустрии}
        {АБ-тесты, SQL, ML модели, визуализация, мобильная аналитика}
    
    \vspace{.1cm}
    
    \entry {2021}
    {Coursera\\
    Введение в машинное обучение}
    {Линейные модели, метрические методы, решающие деревья, градиентный бустинг, кластеризация, PCA}
      
    \vspace{.1cm}
    
    \entry {2021}
    {Stepik\\
    Основы статистики}
    {Статистические тесты, доверительные интервалы, линейная регрессия}
    
    \vspace{.1cm}
    
    \entry {2020}
    {Институт биоинформатики\\
    Летняя школа по биоинформатике}
    {Алгоритмы и ML в биоинформатике}
    
    \section{Достижения}
    
    \entry {2021}
        {Олимпиада студентов и выпускников <<Высшая лига>>\\
        Прикладная математика и информатика}
        {Диплом III степени}
        
    \vspace{.1cm}
    
    \entry {2019}
    {Победитель и призер олимпиад школьников по математике}
    {РЭ ВОШ, Ломоносов, Физтех}

    \section{Проекты}
        
    \entry {\interval{2020}{2021}}
    {IQ-Beat\\
    Стажер в проекте по анализу ЭКГ сигналов}
    {Реализация пайплайна предобработки ЭКГ сигнала на Python3 (numpy, scipy, pandas)} 
    
    \vspace{.1cm}
        
    \entry {2019}
    {Московская школа программистов Яндекс\\
    Веб-приложение агрегатор преподавателей}
    {Реализация веб-сервиса на Python3 (Django, sqlite3)} 

    \section{Навыки}
    	\begin{tabular}{ >{\bfseries}r | l }
    		Языки программирования & Python3, C/C\texttt{++}, SQL, Bash, assembly \\
    		Технологии & RabbitMQ, gRPC, Greenplum  \\
    		Фреймворки и библиотеки & sklearn, pytorch, pyspark, numpy, pandas, matplotlib  \\
    		Инструменты & Linux, Git, Jupyter notebook, gdb, Zeppelin, Tableau \\
    		Теория & \specialcell{Машинное обучение, глубинное обучение, прикладная статистика,\\распределенные системы, алгоритмы и структуры данных} \\
    		Языки & Русский (родной), English (intermediate)
    	\end{tabular} 
\end{document}